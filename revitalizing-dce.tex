% This is "sig-alternate.tex" V2.0 May 2012
% This file should be compiled with V2.5 of "sig-alternate.cls" May 2012
%
% This example file demonstrates the use of the 'sig-alternate.cls'
% V2.5 LaTeX2e document class file. It is for those submitting
% articles to ACM Conference Proceedings WHO DO NOT WISH TO
% STRICTLY ADHERE TO THE SIGS (PUBS-BOARD-ENDORSED) STYLE.
% The 'sig-alternate.cls' file will produce a similar-looking,
% albeit, 'tighter' paper resulting in, invariably, fewer pages.
%
% ----------------------------------------------------------------------------------------------------------------
% This .tex file (and associated .cls V2.5) produces:
%       1) The Permission Statement
%       2) The Conference (location) Info information
%       3) The Copyright Line with ACM data
%       4) NO page numbers
%
% as against the acm_proc_article-sp.cls file which
% DOES NOT produce 1) thru' 3) above.
%
% Using 'sig-alternate.cls' you have control, however, from within
% the source .tex file, over both the CopyrightYear
% (defaulted to 200X) and the ACM Copyright Data
% (defaulted to X-XXXXX-XX-X/XX/XX).
% e.g.
% \CopyrightYear{2007} will cause 2007 to appear in the copyright line.
% \crdata{0-12345-67-8/90/12} will cause 0-12345-67-8/90/12 to appear in the copyright line.
%
% ---------------------------------------------------------------------------------------------------------------
% This .tex source is an example which *does* use
% the .bib file (from which the .bbl file % is produced).
% REMEMBER HOWEVER: After having produced the .bbl file,
% and prior to final submission, you *NEED* to 'insert'
% your .bbl file into your source .tex file so as to provide
% ONE 'self-contained' source file.
%
% ================= IF YOU HAVE QUESTIONS =======================
% Questions regarding the SIGS styles, SIGS policies and
% procedures, Conferences etc. should be sent to
% Adrienne Griscti (griscti@acm.org)
%
% Technical questions _only_ to
% Gerald Murray (murray@hq.acm.org)
% ===============================================================
%
% For tracking purposes - this is V2.0 - May 2012

\documentclass{sig-alternate}
\usepackage{array}

\usepackage{xcolor}
\usepackage{listings}

\lstdefinestyle{CStyle}{
    basicstyle=\footnotesize,
    breakatwhitespace=false,         
    breaklines=true,                 
    captionpos=b,                    
    keepspaces=true,                 
    numbers=left,                    
    numbersep=5pt,                  
    showspaces=false,                
    showstringspaces=false,
    showtabs=false,                  
    tabsize=2,
    language=C
}

\begin{document}
%
% --- Author Metadata here ---
\conferenceinfo{WOODSTOCK}{'97 El Paso, Texas USA}
%\CopyrightYear{2007} % Allows default copyright year (20XX) to be over-ridden - IF NEED BE.
%\crdata{0-12345-67-8/90/01}  % Allows default copyright data (0-89791-88-6/97/05) to be over-ridden - IF NEED BE.
% --- End of Author Metadata ---

\title{Revitalizing DCE}

\subtitle{[Extended Abstract]
\titlenote{A full version of this paper is available as
\textit{Author's Guide to Preparing ACM SIG Proceedings Using
\LaTeX$2_\epsilon$\ and BibTeX} at
\texttt{www.acm.org/eaddress.htm}}}
%
% You need the command \numberofauthors to handle the 'placement
% and alignment' of the authors beneath the title.
%
% For aesthetic reasons, we recommend 'three authors at a time'
% i.e. three 'name/affiliation blocks' be placed beneath the title.
%
% NOTE: You are NOT restricted in how many 'rows' of
% "name/affiliations" may appear. We just ask that you restrict
% the number of 'columns' to three.
%
% Because of the available 'opening page real-estate'
% we ask you to refrain from putting more than six authors
% (two rows with three columns) beneath the article title.
% More than six makes the first-page appear very cluttered indeed.
%
% Use the \alignauthor commands to handle the names
% and affiliations for an 'aesthetic maximum' of six authors.
% Add names, affiliations, addresses for
% the seventh etc. author(s) as the argument for the
% \additionalauthors command.
% These 'additional authors' will be output/set for you
% without further effort on your part as the last section in
% the body of your article BEFORE References or any Appendices.

\numberofauthors{1} %  in this sample file, there are a *total*
% of EIGHT authors. SIX appear on the 'first-page' (for formatting
% reasons) and the remaining two appear in the \additionalauthors section.
%
\author{
% You can go ahead and credit any number of authors here,
% e.g. one 'row of three' or two rows (consisting of one row of three
% and a second row of one, two or three).
%
% The command \alignauthor (no curly braces needed) should
% precede each author name, affiliation/snail-mail address and
% e-mail address. Additionally, tag each line of
% affiliation/address with \affaddr, and tag the
% e-mail address with \email.
%
% 1st. author
\alignauthor
AuthorName\\
       \affaddr{Institute for Clarity in Documentation}\\
       \email{mail@mail.com}
}
% There's nothing stopping you putting the seventh, eighth, etc.
% author on the opening page (as the 'third row') but we ask,
% for aesthetic reasons that you place these 'additional authors'
% in the \additional authors block, viz.

% Just remember to make sure that the TOTAL number of authors
% is the number that will appear on the first page PLUS the
% number that will appear in the \additionalauthors section.

\maketitle
\begin{abstract}
%  * What was done? 
This paper describes the design, implementation and validation
of the ns-3 model of the Licklider Transmission Protocol, the standard
transport protocol used to provide transmission reliability in Delay Tolerant Networks (DTNs).
%  * Why do it? 
DTNs are an emerging field whose principles are used to provide communications 
in extreme and performance-challenged environments, such as spacrecraft,underwater, or
disaster response scenarios. Evaluation of such environments requires the use of simulation tools.
As of now, there is a lack of precise simulation models of these protocols, and concretely within the ns-3 simulator.
%  * What were the results?
The ns-3 model presented in this paper accurately models the LTP protocol and offers ...
\end{abstract}

% A category with the (minimum) three required fields
\category{C.2.2}{ Computer-Communication Networks }{Network Protocols} [Protocol architecture]
%A category including the fourth, optional field follows...
\category{I.6.5}{ Simulation and Modeling }{ Model Development}

\terms{Theory}

\keywords{ACM proceedings, \LaTeX, text tagging}

\section{Introduction}

The rest of this paper is organized as follows: Section 2 provides an overview on Delay Tolerant
Networks and its transmission protocol standards. Sections 3 describes the design and implementation of the Licklider
Transmission Protocol ns-3 module. Section 4 presents the testing approach procedure. Section 5 shows
the validation procedure for achieving interoperability against existing implementations. Lastly, section 6 offers the conclusions and future work.


\section{Challenges}


\subsection{libio vtable mangling}
The vtable is abstractly a table maintainig references to functions called for virtual functions defined for a class or an entity. These functions can 
be overriden dynamically by user defined functions and the respective call for the cooresponding virtual function in a derived clas object can be bound 
to that function at runtime, unlike pre-defined functions which are static and fixed, and can be determined during compile time. The libc on Linux provides
this highly flexible feature for all other user defined classes but the case with the FILE structure is not the same.


The FILE structure is a library defined structure which defines the overall organization, orientation and properties of any file I/O stream opened
by the host application. It maintains different parameters to store useful operational fields like the UNIX based file descriptor number of the 
opened stream, the read/write offsets and buffer addresses of the stream. The pseudoname for the FILE structure as seen inside libc is \textit{\_IO\_FILE}. 
Since, FILE is a library defined entity, the library provides it's own set of implementation for all possible operations on an open FILE stream.
Whenever an \textit{\_IO\_FILE} stream is allocated by the kernel, a contiguous memory location is allocated as a block called \textit{\_IO\_FILE\_plus}. 
The \textit{\_IO\_FILE\_plus} structure looks like this.

\begin{lstlisting}[style=CStyle]     
struct _IO_FILE_plus
{
  FILE file;
  const struct _IO_jump_t *vtable;
};
\end{lstlisting}

Now, by nature of implementation of the kernel's memory allocation processes, the contents of a struct are allocated in contiguous memoory locations. 
This can be veriied by the \textit{sizeof} operation of C to verify that the sum of sizes of the individual members of a struct is equal to the size of
the struct object. Similarly, the FILE and the \textit{\_IO\_jump\_t} objects are allocated in contiguous memory locations. Specifically, the 
\textit{\_IO\_jump\_t} areas is interesting to us, as it defines the callbacks or refference pointers to the functions handlers for each supported file 
operation. Some of the callback which are interesting to DCE and it's use cases are highlighed below.

\begin{lstlisting}[style=CStyle]     
struct _IO_jump_t
{
...
ssize_t(*) __read (FILE *, void *, ssize_t);
ssize_t(*) __write (FILE *, const void *, ssize_t);
off64_t (*) __seek (FILE *, off64_t, int);
int (*) __close (FILE *);
int (*) __stat (FILE *, void *);
...
};
\end{lstlisting}

This structure acts like the vtable for the FILE structure, but it does not behave like the oridinary vtable seen when working with virtual functions 
and derived classes, which are dynamic and supports run time bindings. This vtable is rather expected to behave as a statically bound vtable (there does 
exist other libc functions like \textit{fopencookie}, to override some of the FILE operation implementations, but not all, and it also does not atach 
itself to a standard file I/O stream, and rather works with user defined buffers a.k.a cookies). 

DCE, which is supports simulating real applications on top of the ns-3 stack and also provides different networking stack choices (Linux, ns-3, FreeBSD),
for the host application being run to get a simulated real world execution environenment and results. To support such an implementation and to sync 
application operations like system calls, file I/O operations, networking system calls, etc. it need to hijack all such calls and re-route it through 
corresponding handlers based on the application logic and simulation script implementation. Considering file I/O operations, DCE needs to have control 
over read/write/close/seek/stat operations of each open file, which requires us to overwrite the vtable handlers with the corresponding handlers defined 
in DCE's stdio definition source files. 

Taking advantage of the contiguous memory allocation of the FILE and \textit{\_IO\_jump\_t}, we can execute a buffer overflow attack on the FILE object 
to overwrite the vtable with our custom vtable definition for all the operation we would want to overwrite. We can make a dummy \textit{\_IO\_FILE\_plus} 
pointer point to the explicitly casted FILE object. \textit{memcpy} the existing vtable to a local copy, modify and overwrite the stream operations with our 
custom written implementation, and then re-point the vtable field of our dummy \textit{\_IO\_FILE\_plus} to the local modified vtable, and now we have 
control over those stream operations, which can now be routed through and and to behave as ns-3 streams, Unix FD streams, etc. based on the type of file 
descriptor that is defined. This is one of the productive uses of an buffer overflow attack to leverage control over FILE streams to regulate stream 
buffer flushing and data redirection, but the same could be used for use cases which might pose as potential security threats, as it lets penetration 
testers to make use of tools like pwntools etc. to gain control over application execution and important run time CPU register values such as the 
\textit{rip} which is used for the relative addressing of application components(which is also how position-independent-executables work), which is more 
secure as compared to static addressing, where fixed address values of symbols and pointers could be gained by static analysis tools for run 
time application exploitations.

Post libc-2.25, security features have been implemented to glibc to identify such buffer overflow attack. Whenever any FILE operation is executed, glibc 
would verify if the FILE object's vtable could be trusted and is not corruputed or manipulated. To verify this, it makes a call to 
\textit{\_IO\_validate\_vtable}. Every libio vtable is defined in a unique section called \textit{libio\_IO\_vtables}. By definition, libc would trust 
the vtable if the vtable of the current FILE object lies within this section. It checks if the offsets of this vtable lies between 
\textit{\_\_stop\_\_\_libc\_IO\_vtables} and \textit{\_\_start\_\_\_libc\_IO\_vtables}, if it does, we can continue with the operation, if not, libc 
conducts a final check by calling \textit{\_IO\_vtable\_check} which makes final checks on the FILE vtable pointer location, namespace and edge cases
where FILE * objects are passed to threads which are not in the currently linked executable. Since, when we oveflow the \textit{\_IO\_FILE\_plus} and 
overwrite the \textit{\_IO\_jump\_t}, it does not lie in \textit{libio\_IO\_vtables} section and it also does not pass the pointer mangling sanity checks, leading to a \textit{\_\_libc\_fatal (\"Fatal error: glibc detected an invalid stdio handle\");}

\subsection{PIE loading and usage}
PIE or position-independent-executables are applications compiled with special compiler flags, which allow the application to be loaded into random 
memory address, not depending on absolute symbol addresses, avoiding exploits which hijack the call stack by referencing constant function/symbol addresses.
In the case of a PIE, every memory address is accessed with refference to what is called the \textit{\%rip}. The \textit{\%rip} is computed at the time 
of execution when the application is loaded into virtual memory. This makes it difficult for attackers to determine symbol location in memory.

In DCE, we support the execution of real host application in simulation, bridging the networking layer between the host and the specified networking stack,
and also other system calls made by the host applications. Since, we might have to load several applications into memory, and alsp have control over the
position of the \text{main} symbol of the loaded application, we need to have position independent executables, so that when they are loaded into memory,
the symbol positions in memory are dynamic, giving us control over when an application is launched in a simulation which can be configured in the script 
using available ns-3 programming constructs for the \textit{DceApplicationManager} class. To implement this, DCE used the \textit{CoojaLoader} which 
uses dlmopen udner the hood to load the executable into memory. In glibc version newer than 2.25, security checks have been introduced to identify such 
PIE objects being loaded through dlmopen, and in case it finds the \textbf{DF\_1\_PIE} flag in it's ELF Dynamic headers, if would abort with an error.


\subsection{Linux Networking Stack for DCE (LKL vs LibOS) }
Talk about other DTN simulators/modules: the ONE \cite{one} , omnet++ \cite{omnet}, there are also some modules for ns-2 (not totally sure).

This subsection may be worth moving to the introduction as just a simple paragraph

\section{Solutions}
\label{section:design}

\subsection{Custom glibc-2.31 Based Build}

\subsection{Bake Build Automation}

\subsection{Docker environenment for DCE}

\subsection{net-next-nuse-5.10}

Communication between the LTP engine and the Client Service Instance can happen both ways. 
The client service makes requests to the LTP engine (start or cancel transmission) and 
the LTP engine issues back notifications (to report certain events or hand over received data). 
The design of these communications is a local implementation matter, in the ns-3 module:

\begin{itemize}
 \item Requests are provided in the form of API functions ...
 \item Notifications are provided as callback functions ...
\end{itemize}

The LTP may be run at different protocol layers in order to provide support for this
we provide Convergence Layer adapters ...

LTP uses engine IDs as its addressing system, we provide a lookup structure in the form of LTP to IP resolution tables ...


\section{Results}

\subsection{Docker vs Native DCE Simulation Tests}

\subsection{Performance : DCE vs. ns-3}

\subsection{Google BBR v1 Validation Results}

\section{Related Work}

\section{Conclusions}
%\end{document}  % This is where a 'short' article might terminate

%ACKNOWLEDGMENTS are optional
\section{Acknowledgments}
%
% The following two commands are all you need in the
% initial runs of your .tex file to
% produce the bibliography for the citations in your paper.
\bibliographystyle{abbrv}
\bibliography{ltp-ns-3-workshop}  % sigproc.bib is the name of the Bibliography in this case
% You must have a proper ".bib" file
%  and remember to run:
% latex bibtex latex latex
% to resolve all references
%
% ACM needs 'a single self-contained file'!
%
%APPENDICES are optional
%\balancecolumns
%\balancecolumns % GM June 2007
% That's all folks!
\end{document}
